\newpage

\section{TOP SECRET--The Hidden Life of {\tt LattE}}

\subsection{Additional Files Needed}
\begin{tabular}{|l|l|}
\hline
Name & Description\\
\hline 
%\hline
& \\
CompAdjacency & needed for options ``lrs'' and ``vrep''\\
& \\
\hline
& \\
preproc & transforms {\tt Maple} style input to {\tt LattE} input\\
& \\
\hline
& \\
triangulations/ & Subdirectory containing saved triangulations\\
& \\
\hline
\end{tabular}


\subsection{{\tt MAPLE}-style input files}

{\bf Remark: If you use the official {\tt LattE} v1.1 release, you
  need to add ``linearity'' to the returned file if you have
  equations.}  

Admittedly, in some respects the standard {\tt LattE} input file
formats are rather inconvenient.
Suppose, for instance, that we wish to input a system of inequalities
in 30 variables, whereby (in addition to other constraints) the first
three variables must be greater than or equal to 2, and the fourth
variable must be less than or equal to than the sum of the first
three. In a standard {\tt LattE} input file, these four constraints
would be encoded in the following lines:
\begin{verbatim}
-2 1 0 0  0 0 0 0 0 0 0 0 0 0 0 0 0 0 0 0 0 0 0 0 0 0 0 0 0 0 0
-2 0 1 0  0 0 0 0 0 0 0 0 0 0 0 0 0 0 0 0 0 0 0 0 0 0 0 0 0 0 0
-2 0 0 1  0 0 0 0 0 0 0 0 0 0 0 0 0 0 0 0 0 0 0 0 0 0 0 0 0 0 0
 0 1 1 1 -1 0 0 0 0 0 0 0 0 0 0 0 0 0 0 0 0 0 0 0 0 0 0 0 0 0 0
\end{verbatim}
On the other hand, here's how these constraints could be expressed in
a {\tt MAPLE}-style input file:
\begin{verbatim}
x[1] >= 2;
x[2] >= 2;
x[3] >= 2;
x[4] <= x[1] + x[2] + x[3];
\end{verbatim}
For convenience, we have included a utility \textbf{preproc} in the
{\tt LattE} v1.1 release, that converts input files in such
{\tt MAPLE}-style format to {\tt LattE} readable files. To convert a
{\tt MAPLE}-style file named \textit{source} to a {\tt LattE} readable
file named \textit{target}, use the \textbf{preproc} utility within
the {\tt LattE} directory as follows:

\makebox[12 cm]{preproc source target}

The {\tt MAPLE}-style input format is intentionally loose.
Below is a summary of the format and syntax structure:
\begin{itemize}
\item Every line must be terminated with a semicolon character.
\item Each line must contain exactly one relational symbol
(i.e. each line must contain one equation or one inequality).
\item Valid relation symbols are $=$, $>=$, and $<=$
\item A valid expression must appear on each side of a relational symbol.
\item Valid expressions are as such:
	\begin{itemize}
	\item \hspace{1 cm} 3x[1]
	\item \hspace{1 cm} 5
	\item \hspace{1 cm} -2 + 4x[2]
	\item \hspace{1 cm} 4x[3] - 4 + x[2] + 3
	\item \hspace{1 cm} -3x[1] + 2x[10] - x[1]
	\end{itemize}
\item Parentheses are never allowed, nor are any arithmetic operators
  other than + and -
\item The only literals allowed are integer constants, or variables
  with integer coefficients written in the form
$kx[i]$, where k is an integer and i is a positive integer.  The only
  valid variable name is x.
\end{itemize}
Note that variables and constants can appear in any order, on either
side of the relational symbol, and that multiple occurrences of
constants and particular variables are valid within a relation. 

\textbf{EXAMPLE.}
If our input file were as follows,
\begin{verbatim}
x[1] - 3x[3] + 7x[2]  = -2x[1] + 22;
4x[2]                >= 4 - x[1];
x[1] + 2x[3]         <= 6x[1] - 1;
-x[1] - 13           <= 3 + 2x[3];
x[1]                 >= 0;
\end{verbatim}
\textbf{preproc} would convert this file to the following:
\begin{verbatim}
5 4
22 -3 -7  3
-4  1  4  0
-1  5  0 -2
16  1  0  2
 0  1  0  0
linearity 1  1
\end{verbatim}

\subsection{Additional Options}
\begin{itemize}
\item Use ``vrep'' for ``count'' when you want to use vertex
  representation as input. 
\item Use ``lrs'' for ``count'' if you want to use {\tt lrs} instead
  of {\tt cdd} for enumerating vertices.
\item Use ``trisave'' for ``count homog'' or for ``ehrhart'' when you want
  to save the intermediate triangulation.
\item Use ``triload'' for ``count homog'' or for ``ehrhart'' when you want
  to use a previously saved intermediate triangulation.
\item Use ``uni'' for ``count'', ``maximize'', or ``minimize''
when you know that all vertex cones are unimodular.
\item Use ``value'' when you want to use value digging.
\item Use ``single'' when you want to use single cone digging.
\item Use ``gro'' when you want to compute a superset to the Groebner
basis of a toric ideal.
\end{itemize}

\subsection{To Do}
Here are some points to fix in future versions of {\tt LattE}:
\begin{itemize}
\item Fix ``preproc'' to produce correct input files for {\tt
LattE}. (The option ``equ'' does not exist anymore.)
\item Make program ``preproc'' obsolete by introducing a new option
``maple'' that works similarly as option ``cdd''.
\item Rename option to ``value'' for value digging.
\end{itemize}
